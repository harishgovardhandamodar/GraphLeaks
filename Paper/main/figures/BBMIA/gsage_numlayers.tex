

\documentclass{article}

\usepackage{tikz}
\usepackage{tikz}
\usepackage{pgfplots}
\usetikzlibrary{backgrounds, positioning, fit}
\usetikzlibrary{shapes.geometric}
\usetikzlibrary{patterns}

%% put tikzlibrary below if necessary

% set up externalization
\usetikzlibrary{external}
\tikzset{external/system call={latex \tikzexternalcheckshellescape -halt-on-error
-interaction=batchmode -jobname "\image" "\texsource";
dvips -o "\image".ps "\image".dvi;
ps2eps "\image.ps"}}
\tikzexternalize


\begin{filecontents}{gsage.txt}
z   n   pFA pFB
10  2   99.16   70.9 %citeseer
10  4   98.33  56.4
10  6   95.00  54.3
10  8   42.5  32.2
10  16   18.1  16.66
20  {}  0   0 %cora
20  2   97.14   84.5
20  4   97.85  80.3
20  6   90.71  74
20  8   70  49.9
20  16   30.9  29.28
30  {}  0   0  %pubmed
30  2   100   78.3
30  4   100  75.9
30  6   100  75.6
30  8   93.33  66.4
30  16   33.33  18
\end{filecontents}

\begin{document}

\begin{tikzpicture}
    \begin{axis}[
        footnotesize,
        % set the `width' of the plot to the maximum length ...
        width=\textwidth,
        % ... and use half this length for the `height'
        height=0.5\textwidth,
        % use `data' for the positioning of the `xticks' ...
        xtick=data,
        % ... and use table data for labeling the `xticks'
        xticklabels from table={gsage.txt}{n},
        % add extra ticks "at the empty entries to add the vertical lines
        extra x ticks={5,11,17,23},
        % this ticks shouldn't be labeled ...
        extra x tick labels={},
        % ... but grid lines should be drawn without the tick lines
        extra x tick style={
            grid=major,
            major tick length=0pt,
        },
        ymin=0,
        ymax=105,
        xlabel={Number of Layers},
        ylabel={Accuracy \%},
        ylabel style={font=\large},
        % because of the category labels, shift the `xlabel' a bit down
        xlabel style={
            yshift=-4ex,
            font=\large,
        },
        legend style={at={(0.5,-0.4)},anchor=north,legend columns=-1,font=\large},
        area legend,
        % adjust `bar width' so it fits your needs ...
        bar width=8pt,
        % ... and with that you also have to adjust the x limits
        enlarge x limits={abs=1},
        % set `clip mode' to `individual' so the category labels aren't clipped away
        clip mode=individual,
    ]

    % plot the "red" ybars
        \addplot [
            ybar,
            draw=red,
            fill=red!40,
            opacity=0.4,
        ] table [
            % use just the `coordindex' as x coordinate,
            % the correct labeling is done with `xticklabels from table'
            x expr=\coordindex,
            y=pFA,
        ] {gsage.txt};

    % plot the "blue" ybars
        \addplot [
            ybar,
            draw=blue,
            fill=blue!40,
            opacity=0.4,
        ] table [
            x expr=\coordindex,
            y=pFB,
        ] {gsage.txt};

    \addplot[draw=black,mark=*,thick,smooth] coordinates {(0,82.25) (1,76.11) (2,72.58) (3,59.51) (4,55.2)};
    \addplot[draw=black,mark=*,thick,smooth] coordinates {(6,69.30) (7,65.19) (8,63.13) (9,60.06) (10,56.19)};
    \addplot[draw=black,mark=*,thick,smooth] coordinates {(12,69.30) (13,65.19) (14,63.13) (15,60.06) (16,56.19)};
    \addplot[draw=black,dashed,thick,smooth] coordinates {(0,50)(1,50)(2,50)(3,50)(4,50)(5,50)(6,50)(7,50)(8,50)(9,50)(10,50)(11,50)(12,50)(13,50)(14,50)(15,50)(16,50)};


    % add the category labels
        \begin{scope}[
            % because the reference point will be the lower axis line the
            % labels have to be moved a bit more down to don't overlap with
            % the `xticklabels'
            every label/.append style={
                label distance=2ex,
            },
        ]
            \node [label=below:Citeseer] at (axis cs:2,\pgfkeysvalueof{/pgfplots/ymin}) {};
            \node [label=below:Cora] at (axis cs:8,\pgfkeysvalueof{/pgfplots/ymin}) {};
            \node [label=below:PubMed] at (axis cs:14,\pgfkeysvalueof{/pgfplots/ymin}) {};
            
        \end{scope}
\legend{Train Accuracy,Test Accuracy,Inference Accuracy}
    \end{axis}
\end{tikzpicture}


\end{document}
